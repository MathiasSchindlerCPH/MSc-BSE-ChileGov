 \subsection{Relevance}
    %Why is the provided information relevant?
    
    The Government Communication Office is in charge of designing and implementing the communication strategy of the government. For this purpose, a tool that allows to quickly monitor the opinions around a given topic is useful. It is well-known that Twitter conversations are biased and the individuals that interact on the platform are not representative of the whole population, so our tool is not pretending to representing the whole population. But being conscious of the bias of the platform, tracking Twitter conversations can help better design and target communication strategies by monitoring opinions, realizing the agendas that left-leaning and right-leaning users are trying to push and knowing the most influential users. Tracking these metrics can also help to realize increasing concerns of  Twitter users.
    
    
    %, monitoring the opinions, looking what right wing and left wing people is trying to push in agenda, knowing who are the most influential people in this particular social media platform could be useful to design communication strategies, to be aware of possible increasing concerns of the users and to understand better the discussion in the platform.
    
    Looking at our particular example, but having in mind that this methodology is extensible for other topics and time periods we will show how the information could be useful for office work.
    
    First, when a specific event occurs that is related to the topic of interest, we can observe unusual peaks in Twitter activity. In this case, our tool can be useful to raise alerts about unusual activity around certain topics. This can focus the attention of the Government and they can decide whether they want to analyze the topic more in-depth. Governments pay attention to many political topics simultaneously, so knowing that a certain topic has unusual Twitter activity is a helpful to know what issues need special attention. 
    
    Also, tracking most-used words, hashtags and bigrams can give governments insights to what is talked about within a given topic. For instance, realizing that the hashtag \texttt{\#iquique} is trending, can reflect that something is happening in this city. Also the prominence of bigrams like {\it 'inmigrantes, ilegales'} (Eng: 'immigrants, ilegals') over other bigrams indicates that the irregular situation of migrants is a prominent issue. So, if the Government wants to communicate a set of new policies related with immigration, they should probably emphasize the ones that are related with illegal immigration and the control thereof -- at least in their social media strategy.
    
    Distinguishing left-leaning and right-leaning users is also useful to know who is pushing which agenda within certain topics. Governments can have different communication strategies: sometime they want to send messages to their supporters, sometimes to those that oppose them and sometimes they want to send general messages. Therefore for the current Government, knowing the discourse of their supporters (left-leaning) and the opposition's supporters (right-leaning) can be useful for the rhetorical targeting of the Government's messages, depending on who is the objective audience.
    
    Finally, network analysis is useful to understand who are the most influential people in this conversation. In our particular analysis, it is useful to know that Kast is the most important user for the topic of immigration so it is a topic where the oppositions seems to set the tone. Hence, if the Government wants to announce new immigration policies, they should find a way to recover this control. One way could be to look at the network metrics and analyze which influential users are close to the Government's political position and then increase interaction with these. Some of these influential users might well be media companies such as T13 or Cooperativa, and it is useful to know that these would be relevant platforms to present policies on to increase foothold in a given topic where the opposition has more influence.
    
    %is useful to know that it is a topic where the oppositions to the government have higher control, so if you want to announce a new immigration policy, first it could be a good idea to find a way to recover this control. A way to do this also comes from network analysis. Looking at influential nodes that are closer to your political position and increasing the interaction between authorities and these accounts would give more control over the topic. Also, looking at the influential media (as T13 or Cooperativa) could be useful to target where you will announce policies related to the topic. 
  
    
    