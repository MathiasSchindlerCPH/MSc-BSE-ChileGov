\newpage
\clearpage 
\pagenumbering{Roman} %<- roman page numbers for summary

\section*{Summary for the Chilean Communication Office}\label{sec_summ}

%\thispagestyle{empty} %<- no page number

\addcontentsline{toc}{section}{\nameref{sec_summ}} %<- adds to toc despite being unnumbered

    %\epigraph{\textcolor{blue}{''The challenge teams should have a two-page summary of the challenge question, methodology summary, and/or main result in a figure or table. This should focus the team on what you want to communicate. This should also be posted to GitHub as an introduction to the project.''}}{\textcolor{blue}{--- Hannes' email from Jun 15, 12:47 AM}}
    
    %\textcolor{red}{\it Here mention exactly \underline{how} a practitioner can use our tool to analyze a different topic. What {\it exactly} is it they need to change? What can they do with it? How long does it take and how often can they update the data?} 
    
    
    \paragraph{Challenge Task}
        We were asked to build a general-purpose corpus construction methodology for given political topics and apply it to immigration in Chile. We were asked to label users by political affiliation and provide results about talking points and network analysis.
        
        %{\it Lorem ipsum dolor sit amet, consectetuer adipiscing elit. Ut purus elit, vestibulum ut, placerat ac, adipiscing vitae, felis.}
    
    \paragraph{Relevance}
        The Twittersphere is not representative of the whole population, so our social listening tool is not meant to replace traditional telephone- and survey-based opinion polls. It can however complement these methods at a higher frequency and at lower cost.
        Thereby it can help gauge concerns and agendas of this subset of the population and, in this way, be useful for the Chilean Communication Office which is in charge of designing the government's communication strategy. 
        
        Our tool can raise alerts about unusual activity following unusual peaks in Twitter activity after specific events. Governments pay attention to many political topics simultaneously, so knowing that a certain topic has unusual Twitter activity is helpful in regards to what issues need special attention. 
        
        Tracking the metrics presented in this thesis can give the Government insights into the content of tweets and thereby identify specific political agendas. Network analysis is useful when it comes to identifying the most influential users, while distinguishing left- and right-leaning users is helpful to find which political affiliation is pushing which agendas within certain topics. Analyzing these metrics (e.g. using our interactive dashboard) can help the CCO better shape and target their rhetorical presentation of new policies. 
        
        Our methodology takes approximately two days of computational execution time (depending on the amount of tweets to download) and can hence be updated %on a daily basis for analyses. 
        every other day. It requires some manual input for the initial execution, but for subsequent updates it can run autonomously (besides specifying new time periods).
    
    
    
    %The Government Communication Office is in charge of designing and implementing the communication strategy of the government. For this purpose, a tool that allows to quickly monitor the opinions around a given topic is useful. It is well-known that Twitter conversations are biased and the individuals that interact on the platform are not representative of the whole population, so our tool is not pretending to representing the whole population. But being conscious of the bias of the platform, tracking Twitter conversations can help better design and target communication strategies by monitoring opinions, realizing the agendas that left-leaning and right-leaning users are trying to push and knowing the most influential users. Tracking these metrics can also help to realize increasing concerns of  Twitter users.
    
    
    %, monitoring the opinions, looking what right wing and left wing people is trying to push in agenda, knowing who are the most influential people in this particular social media platform could be useful to design communication strategies, to be aware of possible increasing concerns of the users and to understand better the discussion in the platform.
    
    %Looking at our particular example, but having in mind that this methodology is extensible for other topics and time periods we will show how the information could be useful for office work.
    
    %First, when a specific event occurs that is related to the topic of interest, we can observe unusual peaks in Twitter activity. In this case, our tool can be useful to raise alerts about unusual activity around certain topics. This can focus the attention of the Government and they can decide whether they want to analyze the topic more in-depth. Governments pay attention to many political topics simultaneously, so knowing that a certain topic has unusual Twitter activity is a helpful to know what issues need special attention. 
    
    %Also, tracking most-used words, hashtags and bigrams can give governments insights to what is talked about within a given topic. For instance, realizing that the hashtag \texttt{\#iquique} is trending, can reflect that something is happening in this city. Also the prominence of bigrams like {\it 'inmigrantes, ilegales'} (Eng: 'immigrants, ilegals') over other bigrams indicates that the irregular situation of migrants is a prominent issue. So, if the Government wants to communicate a set of new policies related with immigration, they should probably emphasize the ones that are related with illegal immigration and the control thereof -- at least in their social media strategy.
    
    %Distinguishing left-leaning and right-leaning users is also useful to know who is pushing which agenda within certain topics. Governments can have different communication strategies: sometime they want to send messages to their supporters, sometimes to those that oppose them and sometimes they want to send general messages. Therefore for the current Government, knowing the discourse of their supporters (left-leaning) and the opposition's supporters (right-leaning) can be useful for the rhetorical targeting of the Government's messages, depending on who is the objective audience.
    
    %Finally, network analysis is useful to understand who are the most influential people in this conversation. In our particular analysis, it is useful to know that Kast is the most important user for the topic of immigration so it is a topic where the oppositions seems to set the tone. Hence, if the Government wants to announce new immigration policies, they should find a way to recover this control. One way could be to look at the network metrics and analyze which influential users are close to the Government's political position and then increase interaction with these. Some of these influential users might well be media companies such as T13 or Cooperativa, and it is useful to know that these would be relevant platforms to present policies on to increase foothold in a given topic where the opposition has more influence.
    
    %is useful to know that it is a topic where the oppositions to the government have higher control, so if you want to announce a new immigration policy, first it could be a good idea to find a way to recover this control. A way to do this also comes from network analysis. Looking at influential nodes that are closer to your political position and increasing the interaction between authorities and these accounts would give more control over the topic. Also, looking at the influential media (as T13 or Cooperativa) could be useful to target where you will announce policies related to the topic. 
  
    
    
    
    \paragraph{Methodology}
    We develop a methodology that, following 8 steps and with minimal manual input, allows the user to obtain most of the tweets about a given topic in a given country for a given period of time. The corpus from the methodology contains information about the political affiliation of the Twitter users, distinguishing them between left- and right-leaning. Finally, the methodology also allows practitioners to obtain the network of retweets between these users in the selected topic.
    
    Considering that one of the scopes of our challenge is to provide a social listening tool that helps to monitor the conversation on time, we also provide scripts to subsequently update the data for new time periods. 
    
    We analyze the corpus using a custom-tailored Python library considering simple textual measures such as word clouds, bigrams and hashtags. The interaction of Twitter users is analyzed using methods from networks analysis utilizing influence measures such as degree and centrality and interconnection measures such as density and reciprocity.

        
    
    \paragraph{Main Results}
        We find that our methodology works well in constructing a corpus with tweets about immigration, written by Chilean national Twitter users or authors in Chile. It also seems to work for feminism and can hence be utilized for many separate political topics. 
        
        Regarding immigration in Chile from November 2020 to April 2022 we find that right-leaning Twitter users are more active, more influential and more interconnected than left-leaning users. Right-leaning users are mostly concerned with undocumented immigration and lately begin to relate immigration with crime. They also attempt to link immigration with terrorism. Left-leaning users are worried about these rising sentiments which they view as an expression of xenophobia and racism.
        
        The current President is not particularly influential in the Chilean Twittersphere regarding immigration. The previous right-wing candidate José Kast was the most influential, as were some media outlets such as T13, Cooperativa and Biobio. This might be a challenge for the current government communication office considering that until April 2022, the conversation was mainly dominated by their opponents.
    
    
    \paragraph{Recommendations and Limitations}
        When the Government presents new immigration policies, they should emphasize their approach to the undocumented situation of migrants as well as crime when targeting a right-leaning audience. When targeting a left-leaning audience, they should emphasize that the policies consider immigrants rights and that they try to face the increase in anti-immigrant discourse. The Government should also prioritize to announce immigration policies on media outlets that are more influential (T13, Cooperativa and BioBio). In this way, the President %(or a specific Minister) 
        might become more influential in the Chilean Twittersphere on immigration, as we have shown José Kast was substantially more influential in our studied time frame.
        
        The Government should always keep in mind the bias in Twitter data that over-represent opinions from more political active people (despite our results being consistent with previous studies). They should always consider complementing findings with other studies such as opinion polls, focus groups, etc.
    
    
    \paragraph{Improvement Suggestions}
        We suggest to prioritize improvement efforts on classification of political affiliation and bot detection. For more advanced results, to prioritize implementation of sentiment scores and topic modeling algorithms. Finally, the CCO can target efforts towards building a high-quality interactive dashboard as well as constructing more covariates such as gender or age.
        
        
    
        %{\it Lorem ipsum dolor sit amet, consectetuer adipiscing elit. Ut purus elit, vestibulum ut, placerat ac, adipiscing vitae, felis.}
    
    %\vspace{1em}
    %Our methodology takes a day of computational execution time and can hence be employed on a daily basis for analyses. It needs manual input in area X, Y, Z which do not necessarily have to be re-inputted on every daily iteration. 
    
    %\thispagestyle{empty} %<- no page number
    
    \newpage