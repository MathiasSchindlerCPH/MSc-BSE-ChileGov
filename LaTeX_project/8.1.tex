    \subsection{Discussion of Results}
        The main findings from the exploratory text analysis of our corpus in Section \ref{sec_res} are:
        
        \begin{itemize}
            \item 
            Clearly distinguishable peaks of Twitter activity in February 2021, September 2021 and February 2022. These peaks are related with relevant events of massive deportation and violent anti-immigration protests.
            
            \item 
            Right-leaning Twitter users are more active in the Twittersphere. There 12.05\% more right-leaning users compared to left-leaning users in our corpus. In terms of tweets per user during the studied timeframe, right-leaning users post on average 30.94 tweets while left-leaning users post 13.06 on average.
            
            \item 
            Issues of illegality and the irregular migrant situation appear as the most prominent as measured by top words, bigrams and hashtags. Comparing the use of these terms between left- and right-leaning users, we find indications that right-leaning users are the ones primarily pushing these agendas in the online conversation.
            
            \item 
            Topics related to crime are more prominent since the beginning of 2022, as measured by top words. They are mostly used by right-leaning users in this period. Before January 2022, the words related to crime were equally used by left and right and were less relevant in the whole conversation.
            
            \item 
            During the violent anti-immigration protest in September 2021 (the highest peak of Twitter activity in our studied period), right-leaning users seem to mainly be worried with illegal immigration. Left-leaning users are primarily worried about xenophobia and the burned belongings of immigrants by the protesters. Left-leaning people seemed to be trying to push the agenda of the responsibility of the Piñera government (right-leaning government) on what happned while right-leaning users used the event to push campaign hashtags in favor of their candidate José Kast.
            
            %Looking in detail the highest peak of Twitter activity (the protest in September 2021) we found that right wing people are principally worried about illegal immigration in contrast with left wing people that are more worried about xenophobia and the burned belongings. 
            
            %Also, both left-leaning and right-leaning try to push some political issues on the agenda. Left-leaning people pushed the responsibility of the Piñera government on what was happening and linked the immigration crisis with his visit to Cúcuta. Right-leaning people used this event to push some campaign hashtags in favor of their candidate José Kast.
            
            \item 
            Unlabeled users primarily show similar characteristics to right-leaning users during the September 2021-protest, particularly the prominent use of word illegal. They used words and hashtags to push an anti-UN agenda, which neither is present in the subsample of right- nor left-leaning users, but is most likely connected with right-wing sentiments.
            
            %In the same event, unlabeled people seem to be closer to the right wing. Also they used some words and hashtags to push an anti UN agenda. This topic is not present in right-leaning or left-leaning discourse.
        \end{itemize}
                
                %\paragraph{Network Analysis}
        
        \paragraph{Networks Conclusion}
        
        During the campaign, we found different patterns occurring among left and right users in the network. 
        
        \begin{itemize}
        
        \item We found the right wing candidate (Jose Kast) to be the most influential node of our network. It is not so surprising since the candidate proposed to build a trench in the border to stop the illegal trespassing during the campaign. By contrast, the left-leaning candidate is not as influential, and is far below looking at different measures. It can be useful for the Communication Office to be aware of the influence of it's current president in the topic of interest.
        
        
        \item Generally, we found right-leaning users were the most active in the conversations about immigration. This was expected as during the presidential campaigns, immigration was an important argument from the right. In addition, they are better connected among them compared to the left. Letting us think that they are well organized and share the same opinions in this topic. 
        
        \item When looking at the degree measures, we found some right-wing fake account \footnote{For further research, we could establish rules to automate bots and fake accounts search}. We also found certain medias to be particularly influential. Tracking the relevant users is important, it can be useful to be aware of the opponent opinions as well as knowing which medias have coverage over a particular topic. It could potentially help to target better the public and design more adapted communication strategies.
        
        \end{itemize}
        
        
        
                
            
            
            