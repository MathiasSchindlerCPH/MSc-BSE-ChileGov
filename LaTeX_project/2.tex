\section{Previous Studies}\label{sec_lit}
    
    In recent years, Twitter data has received increased attention in academia. Researchers are attracted to data from the microblogging platform because of its simple structure with short and frequent interactions between users and the easy access to data that the Twitter API provides. \cite{tufekci2014big} and \cite{barbera2015understanding} show that Twitter users are not representative of the whole population and this can potentially lead to biased results. However, as long as researchers are aware of these limitations, analyzing Twitter data can still yield relevant insights. %\cite{altoaimy2018driving} analyzes Twitter conversations regarding women’s right to drive in Saudi Arabia. They analyze which users are more likely to support this right and describe their discourse, despite knowing that Twitter users are not representative of the whole population.
    
        \newline\indent
    The literature around political conversations on Twitter is already vast.
    %Looking more closely at our topic and studies that analyze Twitter conversations and interactions for political communication, we found a large literature that focuses on their analysis in electoral periods.
    \cite{jungherr2016twitter} provides an extensive systematization of the literature around this topic up to 2016. The paper finds that the use of the Twitter API is common in these studies and specifically the use of hashtags (34 of 127 studies analyzed) or keywords (26 of 127 papers) as a criteria for identifying topics of a conversation. These studies select specific events where keywords or hashtags are easy to identify: E.g. \cite{lin2014rising} focus on debates or party conventions that had a commonly known hashtag to identify the event (e.g the hashtag \texttt{\#debate} during presidential debates), while  \cite{himelboim_classifying_2017} select a long list of different topics, where each one is represented by a single word or hashtag (e.g the hashtag \texttt{\#OHSen} used to discuss the Senate race in Ohio or the word ''Autism'' to identify people who talk about this neurodevelopmental condition). We are providing a methodology to collect data about broad topics that need more than a single word of hashtag, in order to retain more information than the previously mentioned one-word/-hashtag studies.\footnote{For instance, if in our case we used only the word ''immigration'' we would lose information about people who do not mention this word, but only mention Venezuelans immigrating to Chile. So, we should add the word “Venezuelans” in order to lose less information.} In general, these methods require domain knowledge for choosing which words or hashtags pertain to a given topic and, to the best of our knowledge, there have not yet been developed computational algorithms that can outsource this task from human input.%\footnote{Neural networks might be able to perform this task when this field matures but as of the time of writing such tools are outside the scope of this thesis.}
    
    \newline\indent
    Recent research with Twitter data include hate speech detection \citep{plaza-del-arco_comparing_2021, basile_semeval_2019, pereira-kohatsu_detecting_2019}, sentiment analysis \citep{agarwal2011sentiment, saif2012semantic} and feature engineering such as extrapolating Twitter user's age, gender and political affiliation \citep{conover_predicting_2011, pennacchiotti2011machine, kruspe_changes_2021}. Identifying Twitter users' political affiliation is particularly relevant for our work because we base our political affiliation identification in part on the methodology developed in \cite{rao2010classifying}, considering the use of hashtag as the main criteria to identify affiliation. Our methodology also contributes to the literature on feature engineering on Twitter data, as we propose a novel methodology to identify Twitter users' nationality. To the best of our knowledge, feature engineering nationality of Twitter users has not been done before, and we believe this to be fruitful in future research across multiple topics utilizing Twitter data.
    
        %\newline\indent
    %The development of the Bidirectional Encoder Representations from Transformers model (also known as \texttt{BERT}) in  \cite{devlin2018bert} has proven very influential in text analysis. The model was trained using the big data set of \texttt{BooksCorpus} and English Wikipedia. The methods and outputs that the model provides has led to significant improvements for various classification tasks in text analysis and also provides a powerful tool to work with text as data. The methods that they used caught the attention of many researchers, who started to work in similar models but trained these with different data sets. Twitter and Spanish language was not an exception, and researchers such as \cite{perez2021robertuito} and \cite{gonzalez2021twilbert} focus their work in training similar models on Spanish Twitter corpora. These tools are useful to analyze Twitter data and for classification tasks, but they are computationally demanding, leading to a trade-off between how quick we need some results and the use of these recently developed tools.
    
        %\newline\indent
    In terms of network analysis, using graph theory can be relevant for analyzing information flows. Twitter data contains various types of interactions, the most prevalent ones being ''retweets'', ''likes'', ''mentions'' and ''follows''. \cite{conover2011political} compares retweets and mentions networks, finding that political retweets exhibit a highly segregated partisan structure. On the other hand, the  network of mentions is dominated by a single politically heterogeneous cluster. %Depending on the question studied, the most appropriate link needs to be chosen. 
    Following these conclusions, for analyzing information diffusion networks and identifying clusters, retweets are most appropriate. %\cite{stewart2018examining}  
    Once the network is correctly built, various metrics can be used to characterize the network. Common metrics include volume, influence and density. \cite{maharani2014degree} use retweets as links between users and identify patterns among influential users using degree and eigenvector centrality. We replicate these methods in the immigration retweets network. %They found eigenvector centrality is a less computationally expensive method compared with degree centrality.
    %{\it((\cite{bramson2017understanding} used 9 different types of polarization such as Distinctness or Group Divergence (distance between groups) and Group Consensus (distance within each group).))}
    
    
    %Finally, reviewing studies that analyze immigration in Chile could be useful to contrast our results. In terms of general perception about immigration, there do not exist many recent studies that analyze the general opinion.
    Studies about Chilean's general perception towards immigration are scarce. One such study is \cite{gonzalez2019como} which analyzes attitudes towards immigration and their relationship with social diversity between 2002 and 2017. Some interesting findings are the differences in attitudes when considering the nationality of immigrants and also that people who self-identify with right-political positions have slightly higher anti-immigrant attitudes compared to people who self-identify with center or left-leaning political positions. The study also shows data from polls that affirm the existence of a growing concern about illegal immigration.
    
    %The two main works related are \cite{gonzalez2019como} and \cite{glvez_2020_barmetro}. The first one analyzes the attitudes towards immigration and their relationship with social diversity, considering the period between 2002 and 2017. Some interesting findings are the difference in attitudes considering the nationality of the immigrant and also that people who self-identified with right political positions have slightly higher anti-immigrant attitudes compared with people self-identified with center or left political positions. Also, they showed some data from polls that reaffirm the existence of a growing concern about illegal immigration.
    
    \cite{glvez_2020_barmetro} uses Chilean Twitter data to analyze discriminatory message against immigrants between January 2018 and August 2020. The study finds that Twitter activity and discriminatory speech are sensitive to public immigration-related events and that discriminatory speech mainly originates from far-right and nationalist users. The study also finds that immigration is used as a topic to attack certain political figures.
    
    
    %The main findings are that the Twitter activity and discriminatory index are sensitive to public events related with  immigrants, that the discriminatory messages are mainly from far-right wing users, nationalist and associated with the alternative “Rejection” in the National Referendum celebrated in September 2020. Also the study found that immigration is used as a topic to attack certain political figures and the main focus of the message was to support specific political positions. The discriminatory messages are related with a defense of Piñera Government and with attacks to the previous administration.

        

