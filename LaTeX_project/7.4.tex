 \subsection{Unlabeled User Accounts}\label{unlab_users_affiliation}    
        
        To further analyze the characteristics of the unlabeled users we can compare the findings from the networks metrics with those from the textual results. 

            \newline\indent
        Figure \ref{fig_bigram_protest_unlab} gives insights as to which ideology is most prevalent in the 'Unlabeled' category. The distribution of bigrams is more akin to that of the right-leaning users and so are the connotations of the bigrams. We find unlabeled users to mainly stress the undocumented/illegal situation of the migrants with popular bigrams such as {\it 'inmigrantes, ilegales'} (Eng: Immigrants, illegals), {\it 'crisis, migratoria'} (Eng: Crisis, migratory), {\it 'marcha, encontra'} (Eng: March, against) and {\it 'encontra, migrantes'} (Eng: Against, migrants). So, in the unlabeled category we find that these are more similar to right-leaning users. It is possible that we have some right-leaning users in the 'Unlabeled' category that our labeling strategy classifies wrongly. It could also be the case that unlabeled users are in fact center-leaning and that center voters are more anti-immigration than embracing. This supports the finding from Figure \ref{fig_entire_graph_1000} that unlabeled users are mostly centered close to right-leaning users.
    
            \newline\indent
        From Figure \ref{fig_hashtags_protest_unlab}, we also find frequent usage of hashtags such as \texttt{\#nomasinmigrantes} (Eng: No more immgirants) and \texttt{\#nomasinmigrantesilegales} (Eng: No more illegal immigrants) which mainly mirror the talking points of the right. However, we find one specific talking point from the left-leaning users which is \texttt{\#xenofobia} (Eng: Xenophobia). With these findings in mind, we cannot claim that unlabeled users primarily are right-leaning as the use of hashtags is ambiguous across ideological agendas. This is again consistent with findings from Figure \ref{fig_entire_graph_1000} that unlabeled users seem most similar to right-leaning users, but that there is a minority more similar to left-leaning users.
        
            \newline\indent
        While analyzing the unlabeled users, we also discover a new pattern. In Figure \ref{fig_bigram_protest_unlab} the mention of \texttt{@onuchile} (UN in Chile's account) is prevalent, while Figure \ref{fig_hashtags_protest_unlab} shows that hashtags such as \texttt{\#fueraonu} (Eng: Out with the UN) and \texttt{\#nomasonu} (Eng: No more UN) are popular among unlabeled Twitter users. These bigrams and hashtags are neither used by left-leaning nor right-leaning users. From Appendix Tables \ref{apptab_unlabeled_bigrams_protest} and \ref{apptab_unlabeled_hashtags_protest}, we find that it is only a few number of users that tweet these hashtags and bigrams while making heavy use of them. 
        
            \newline\indent
        Generally, it seems that unlabeled users are primarily reminiscent of the right-leaning users. 
        %and if we assume anti-UN sentiments to be right-wing agendas, it could be a reasonable assumption that unlabeled users are primarily right-leaning. In this case, the previous findings, e.g. those from Figure \ref{fig_terms}, could potentially be even starker. 
        More accurately categorizing the unlabeled users is therefore one of the most immediate future improvements to our project, as is further discussed in Section \ref{sec_disc_improv}. Reviewing how many users include these hashtags or bigrams, we can see that there are only a few of them that tweet a lot. Again, we find that something that appears to be a general pattern in truth is a little group of people trying to push some topic in the discussion.