\section{Results}\label{sec_res}
    This section presents how our developed methodology of corpus construction can be utilized by political actors in practice to gauge public opinion. Because Chile has recently seen an uptick in immigrants (375,388 arrivals in 2010 \citep{datos_2010}, compared to 1,462,103 in 2020 \citep{ine_2020}) and multiple violent anti-immigration protests in recent years, it could be hypothesized that anti-immigration sentiment is on the rise in Chile. Further, it might be the case that right-leaning users hold stronger anti-immigration views while left-leaning users are more embracing of immigration. Our social listening tool can then be used to analyze whether this is the case in Chileans' online conversations about immigration.
    
    As a general political context of the analyzed period, the previous Chilean government was right-wing. %, led by Sebastián Piñera. 
    The most recent election was held on December 19\textsuperscript{th}, 2021, and elected Gabriel Boric as President of Chile. %For the first time since 1990, the two candidates who reached the second round of the presidential election, were not affiliated with the traditional political parties. 
    Boric is affiliated with a leftist party while the other presidential candidate in the second round was José Kast, who is affiliated with a far-right party. Immigration was one of the relevant topics discussed during the campaign, particularly pushed by the candidate Kast. Boric took office on March 11\textsuperscript{th}, 2022.
    
    The corpus in this section is constructed by applying the methodology in Section \ref{sec_meth_gene} with the following characteristics:
        \begin{itemize}
            \item 
            Country: Chile
            
            \item 
            Topic: Immigration
            
            \item
            Time frame: November 1\textsuperscript{st}, 2020 to April 11\textsuperscript{th}, 2022
        \end{itemize}
                
        The time frame is chosen to include periods before and after the last presidential election and three significant local events regarding immigration: One mass deportation of immigrants on February 10\textsuperscript{th}, 2021 and two violent anti-immigration protests in northern border cities on September 26\textsuperscript{th}, 2021 and January 29\textsuperscript{th}, 2022, respectively. A detailed description of how we applied our methodology to immigration in Chile can be found in Appendix \ref{app_sec_meth}.