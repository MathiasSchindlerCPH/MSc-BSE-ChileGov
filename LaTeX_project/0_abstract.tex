\begin{abstract}
    
    \noindent
        This thesis presents a general-purpose corpus construction methodology with Twitter data for a given political topic in a given country. It applies the methodology to immigration in Chile from November 2021 to April 2022, resulting in a corpus with 573,999 tweets. Our results indicate increasing anti-immigration views from Chilean Twitter users. Right-leaning users are more active and more anti-immigration. Left-leaning users are mostly concerned with xenophobia and racism. Utilizing network analysis methods, we find that right-leaning users are also more influential and interconnected. The results are consistent with previous studies and the methodology is robust to other political topics such as feminism. 
        
            \newline\indent
        Future improvements could include more advanced classification algorithms for political affiliation and bot detection. Practitioners using our social listening tool should be aware of the general misrepresentation of Twitter users in regards to a general population.
        
        
        
    \vspace{1cm}
    \newline\noindent
        {\bf Keywords:} 
        Twitter, Corpus Construction, Politics, Network Analysis, Reproducibility, Chile, Immigration
        
    %\vspace{0.15cm}
    %\newline\noindent
    %    {\bf JEL Codes:} 
    %    O54, F5, P16, Y8
\end{abstract}