\section{Conclusion}\label{sec_conc}
    This project has shown that our methodology works as intended to construct a corpus with political topic-specific Twitter conversations. It has been shown to work well across multiple topics and provide relevant descriptive insights regarding users' concerns and agendas and the structure of their interactions. It is important for practitioners to be aware of the project’s limitations, primarily that Twitter users do not represent the whole population and that extreme opinions hence might be over-represented. %It should also be noted that our social listening tool is yet to be tested on more topics and for other countries than Chile.
    
    We believe this project to be immensely useful for governments as it can provide opinion data at high frequency and at low cost and thereby complement traditional phone- and survey-based opinion polls. Insights from the data can help political institutions realize important topics and agendas within topics across political affiliations. Metrics from network analysis can help identify the most influential Twitter users. Insights can be easily obtained by visualizing the data as an interactive dashboard. %Such information can help political institutions better design and target communication strategies.
    Our entire methodology is available in our GitHub repository and is structured with ease-of-usage as a main priority.

    This thesis presented the first version of our social listening tool. As future improvements we propose to add more advanced classification algorithms for political affiliation and bot detection. To provide more precise descriptions of users' concerns and agendas, we further propose to analyze sentiment scores and topic modeling and to distinguish subgroups by constructing features such as users' gender and age.
    
    While applying the social listening tool we obtained novel findings about Chilean Twitter conversations regarding immigration. By analyzing a corpus of Chilean Twitter conversation about immigration from November 2020 to April 2022, we found the Chilean Twittersphere to become increasingly concerned about immigration. This is especially pronounced for right-leaning users. These post up to thrice as often as left-leaning users and retweet more often. We find that right-leaning users are primarily concerned about the undocumented situation of immigrants and crime. They also exploit specific events of higher general Twitter activity, such as a violent protest in September 2021, to campaign for the politicians they support. Right-leaning users also retweet each other more often and are more influential on Twitter -- especially their presidential candidate José Kast. 
    
    Left-leaning users are primarily concerned about their views of rising xenophobia and racism. They do not have substantial influence on Twitter about immigration and are outdominated by the right-leaning users, especially by José Kast. Our findings are consistent with previous studies. 
    
    We also make a particularly surprising finding: Some agendas are prevalent in the general Chilean Twittersphere, but are used only by a few users in high frequency. This indicates that certain users try to aggressively push these agendas into the general conversation. This pertains to two specific agendas: Linking immigration with terrorism and blaming the UN for Chile's migration crisis.
    
    In addition to being useful for practitioners, this thesis also contributes to the academic literature in three ways. First, we provide a general-purpose methodology to accurately construct a corpus of Twitter conversations regarding a specific topic for a given country. Second, we contribute to the field of feature engineering by extrapolating users' geolocation and nationality, which can be useful across virtually all studies using Twitter data. Third, we add to the scarce literature regarding public opinion in Chile towards immigration.
    
    We believe our thesis to have presented new avenues for further research and presented a first prototype for a social listening tool for practitioners.