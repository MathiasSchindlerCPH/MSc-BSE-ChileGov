\clearpage 
\pagenumbering{arabic} %<- start introduction on page no. 1

\section{Introduction}

    Digital communication is becoming increasingly important everyday. In fact, social media platforms such as Twitter can provide information in real-time about the opinions of a population's subgroups. Developing social listening tools that can extract and clean social media data can therefore be helpful for political actors to gauge public opinion regarding particular issues or implemented policies. They can thereby complement traditional telephone- or survey-based opinion polls at higher frequency and lower costs.
    
    The work in this thesis is motivated by a task given to us by the Chilean Communications Office (henceforth, CCO) which was stated as follows
    
    \epigraph{''Develop a methodology to analyze the Chilean conversation on Twitter on a specific topic, considering the political affiliation of the users. Describe the associated narratives that appear over time and the network structure of the groups involved. Apply the developed methodology using immigration as the test subject.''}{--- (See Appendix \ref{app_sec_challenge} for original text)}
    
    %Our contribution to the CCO and the literature in general is two-fold. First, we build a general-purpose Twitter corpus construction methodology with conversations about a given political topic in a specified time frame. Second, we apply this methodology to provide descriptive evidence as to how political affiliation shapes the online conversation about immigration in Chile. Generally we find that over a two-year period until April 2022, the discourse in the Chilean Twittersphere becomes more focused on anti-immigration and that this trend is particularly prevalent among politically right-leaning Twitter users. We also find that right-leaning users are more active, have a stronger community and generally have a higher capacity to push certain talking-points on immigration in Chile. Our findings are consistent with previous studies. 
    
    Our contribution to the CCO and the literature in general is two-fold. First, we build a general-purpose Twitter corpus construction methodology with conversations about a given topic in a specified time frame distinguishing between left- and right- wing users. Second, we apply this methodology to provide descriptive evidence as to how political affiliation shapes the online conversation about immigration in Chile in order to answer the questions: What are the main concerns of Chilean Twitter users regarding immigration? Which are the main differences between left-leaning and right-leaning discourses? Who are the most influential users in this conversation? Generally we find that over a one and a half year period until April 2022, the discourse in the Chilean Twittersphere becomes more focused on anti-immigration and that this trend is particularly prevalent among politically right-leaning Twitter users. We also find that right-leaning users are more active, have a stronger community and generally have a higher capacity to push certain talking-points on immigration in Chile. Our findings are generally consistent with previous studies, while also providing new insights. 
    
    We reach these insights by analyzing a corpus of tweets written by Chilean users from November 2020 to April 2022 consisting of 573,999 tweets. The corpus is built by adapting the general-purpose methodology, specifying keywords and hashtags pertaining to immigration in Chile. To extrapolate meaningful insights from the data, we build a custom Python library and an interactive dashboard tailored to the corpora resulting from our methodology. Utilizing these tools, we analyze metrics such as the most-used words, hashtags and bigrams as well as measures from network analysis. 
    
    First, we analyze the entire time period. By analyzing a political affiliation-labeled subsample of 216,245 tweets, we find that right-leaning users post close to thrice as many posts per user as left-leaning ones. Left-leaning users primarily push anti-xenophobia agendas while right-leaning users mostly talk about undocumented immigrants and blame these for crime. We also find that a small number of right-leaning users who aggressively try to push an agenda relating immigration with terrorism (which is mostly carried out by nationals in Chile) and blame the left for both issues. 
    
    Second, we analyze how conversations were shaped during a violent anti-immigration protest in September 2021. The pattern that right-leaning users mostly talk about illegal immigration and left-leaning users talk mostly about xenophobia and racism is still present in this period. Also we found that right-leaning users used this protest as a pretext to campaign and position their candidate in the discussion. In contrast, left-leaning users did not link the protest with their candidate's campaign and the main political issue that they pushed was to blame the previous right-wing government for the migration crisis. An interesting insight appears while looking at the group of users that are neither classified left- or right-leaning. These raised an anti-UN campaign, linking a UN agenda with the immigration crisis. We again find a small subgroup of users, pushing these agendas aggressively.
    
    Finally, using network metrics, we show that the retweet network of left-leaning and right-leaning users differ substantially: Right-leaning users are more interconnected and more active than left-leaning users. The former right-leaning presidential candidate José Kast was the most influential account while certain media outlets were also influential. These findings are consistent with the text analysis. All together this indicates that right-leaning users dominate the Twitter conversation around immigration in Chile, led by the former candidate Kast. 
    
    We believe our work to be immensely useful for governments and political institutions in general. Despite Twitter users generally not being representative of a whole population, we believe the benefits of quickly monitoring opinions around certain topics at a low cost outweigh the disadvantage of analyzing conversations of a not representative subsample of the whole population. Knowing the most influential accounts as well as the agendas and concerns from both sides of the political spectrum can help inform to design better targeted communication strategies from political institutions.
    
    The rest of this thesis is structured as follows: Section \ref{sec_lit} goes through previous, relevant studies; Section \ref{sec_meth} present our general-purpose methodology; Section \ref{sec_res} presents results regarding immigration in Chile; Section \ref{sec_dis} discusses the results and future improvements. The codes for the social listening tool developed in this thesis are available at our GitHub repository under the following url: \url{https://github.com/BSE-DSDM-2022/ChileGov}.
    
    
    
