\subsection{Validation of Results}\label{sec_disc_prev}
    
    \paragraph{Comparison with Previous Studies}
    In Section \ref{sec_lit}, we reviewed the most relevant, previous studies about public perception of immigration in Chile. Here we compare these findings with our own from Section \ref{sec_res}. Studies that analyze the general perception of immigration in Chile are generally scarce. This is bad for the purpose of comparing our findings but also shows why our tool is relevant -- it provides a description of the conversation around immigration that is not easily accessible today. 
    
    \cite{CEP_86} included some questions about immigration in their latest poll covering April to May 2022. (The CEP is one of the most prestigious opinion research centers in Chile.) In the study, 13\% of respondents mentioned immigration as the main pressing issue for the Government. This contrasts with 6\% in August 2021 and 1\% in December 2019. This is consistent with our findings from Section \ref{sec_res_corp} that Chileans are increasingly concerned about immigration. In a different question, respondents were asked to rank the strictness of their preferred immigration policies on a scale from 1 (most restrictive) to 10 (most lax). 61\% of respondents answered between 1 and 4 (i.e. immigration-skeptic), 30\% answered 5 or 6 (i.e. center-leaning) and 8\% answered 7 to 10 (i.e. immigration-friendly). This is again consistent with our findings, as the most common hashtags and bigrams show opposition towards immigration. This particular question has only been asked in the most recent poll and can hence not be compared over time.
    
    \cite{glvez_2020_barmetro} analyzes Twitter data, focusing mainly on classifying discriminatory messages and their authors from January 2018 to August 2020. The study was sponsored by the Jesuit Service for Migrants (henceforth SJM for {\it Servicio Jesuita a Migrantes}), which is one of the most relevant NGOs working with immigrants in Chile. The study finds that users who use discriminatory language are from the {\it ''political extreme right, nationalist and conservative''} and that they {\it ''declare themselves anti-leftists''}. This is consistent with our findings that right-leaning Twitter users in general hold anti-immigrant positions as seen from hashtags such as \texttt{\#noesimmigracionesinvasion} which could be considered extreme right. Anti-leftist agendas are seen from the trending hashtag \texttt{\#izquierdamiserable} by right-leaning users. %\textcolor{red}{(I want to write something about nationalism – do right-leaning use “patriota” in their bios, i.e. more than left?)} 
    The study also presents a word cloud of the account description of users that use discriminatory language (attached in Appendix Figure \ref{appfig_cloud_SJM} for the reader’s convenience). Comparing with our corresponding Appendix Figure \ref{appfig_auth_desc_cloud}, we find some repeated words like {\it ’Rechazo’} (Eng: Reject) or {\it ’Derecha’} (Eng: Right). Differences do exist due to different sample selection criteria and time frames, but the existence of common patterns is consistent considering that both are analyzing the same topic.
    
    \cite{gonzalez2019como} is an academic study about general perceptions about immigration in Chile from 2002 to 2017. The study presents polling data showing that 57\% of Chileans think that irregular migration is a problem and that the Government should exclude illegal immigrants. The study also finds that right-leaning individuals have slightly higher anti-immigration attitudes compared to individuals with left-leaning or center-aligned political positions. These findings are consistent with what we found for a different period of time. 
    
    It generally appears that our findings are consistent with previous studies, giving us confidence in the performance of our developed methodology. We can filter Twitter conversations by users’ nationality and topics as well as labeling them by political affiliation, and reproduce previous findings as well as find novel insights. Hence, our social listening tool seems to be accurate and useful in the test subject of immigration in Chile.
    
    
    
    \paragraph{Generalizability of Methodology} 
    
    To validate that our corpus construction methodology from Section \ref{sec_meth} functions well across other relevant political topics we have run the methodology for the topic of feminism in Chile from March 9 to March 13, 2022.\footnote{We only consider a small timeframe in order to have quick results.} The results are presented in Appendix \ref{appsec_feminism}. The highest peak of Twitter activity during this period is on March 9, one day after the manifestation of International Women's Day, which seems reasonable. We also find the top hashtag to be \texttt{\#8m}, which refers to March 8. Boric's presidential inauguration ceremony was held on March 11, where we find the second-highest peak of activity. As a result of this, we find bigrams such as {\it ''feminist government''}. Unlike in the topic of immigration, for feminism we find more left-wing users in the conversation. 
    